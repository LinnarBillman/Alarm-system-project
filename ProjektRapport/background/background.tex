The first electronic home security system dates all the way back to the 19th century, to a man named Augustus Russell Pope, an inventor from Boston \cite{securityhistory}. Pope created a battery powered gadget which reacted to the closing of a circuit. Doors and windows were connected to this circuit in parallel, where closing a window or door would complete the circuit and sound the alarm. The alarm was created by striking a brass bell with a hammer which was powered by electro magnetic vibrations. The alarm would not stop simply by closing the once opened window. Pope patented his invention in 1853. Pope's patent was bought in 1857 by Edwin Holmes, a businessman who later became the founder of the first company for electrical alarm systems.

In the 20th century, the alarm systems started to incorporate motions sensors as well as door and window sensors. This is also when alarm systems became a standard feature of building security. In the end of the 20th century, the first wireless alarm system hit the market and revolutionized the alarm technology. 
